\chapter*{\centerline{\bf \Large\MakeUppercase{Lời mở đầu}}}
\addcontentsline{toc}{chapter}{Lời mở đầu}

Điện toán đám mây\footnote{Cloud Computing: \url{https://en.wikipedia.org/wiki/Cloud_computing}} không còn là một khái niệm mới ở Việt Nam cũng như trên thế giới. Mặc dù điện toán đám mây chỉ là một cách khác để cung cấp các tài nguyên máy tính, chứ không phải là một công nghệ mới, nhưng nó đã châm ngòi một cuộc cách mạng trong cách cung cấp thông tin và dịch vụ của các doanh nghiệp - tổ chức trên toàn thế giới.

Sự phát triển nhanh chóng và mạnh mẽ của điện toán đám mây đã đặt ra nhiều thách thức đối với các nhà cung cấp dịch vụ (cloud providers) cũng như các doanh nghiệp, tổ chức sử dụng  nó. Áp lực đó không chỉ là cơ sở hạ tầng (hệ thống máy chủ, hệ thống mạng, trung tâm dữ liệu .v.v.) mà quan trọng hơn đó là yếu tố con người.

Đối với doanh nghiệp, điện đoán đám mây giúp giảm thiểu thời gian và chi phí đưa sản phẩm tới người dùng. Đối với quản trị hệ thống, điện toán đám mây mang lại khả năng mở rộng hệ thống một cách nhanh chóng, nhưng đồng thời nó cũng đặt ra thách thức trong việc quản lý hạ tầng máy chủ. Với số lượng hàng trăm server thì việc phải quản lý thủ công theo cách truyền thống là không hiệu quả và đem lại nguy cơ cao cho hệ thống. Từ yêu cầu thực tế đó, các framework cho việc tự động hóa hệ thống ra đời. Chúng cho phép người quản trị hệ thống có thể quản lý cấu hình của các máy chủ, đồng bộ hóa chúng với nhau, đồng thời đảm bảo việc nó không bị thay đổi sai bởi sự vô tình của con người. Không những thế, chúng còn cho phép người quản trị hệ thống triển khai ứng dụng một cách nhanh chóng tới nhiều môi trường và máy chủ khác nhau.

Các automation framework tiêu biểu được đề cập trong đồ án này bao gồm: Puppet, Chef, Salt và Ansible. Chúng đều là những sản phẩm mạnh mẽ và được sử dụng rộng rãi bởi nhiều doanh nghiệp, tổ chức lớn trên toàn thế giới.

Trong nội dung của đồ án này, em tập trung tìm hiểu cách sử dụng các automation framework và sau đó là ứng dụng chúng để tự động hóa các công việc liên quan đến hệ thống hạ tầng các sản phẩm công nghệ.

Hạ tầng sản phẩm công nghệ ở đây bao gồm:
\begin{itemize}
\item Hạ tầng máy chủ (Linux) cùng các cấu hình; các dịch vụ hệ thống như NTP, SSH, Cron, DHCP, DNS, .v.v.
\item Các dịch vụ và ứng dụng phục vụ trực tiếp cho sản phẩm như webserver (apache, nginx), database (postgresql, mysql) .v.v.
\end{itemize}

Việc phân chia 2 tầng như trên chỉ mang tính chất tương đối để chúng ta có thể xác định cụ thể quyền hạn và trách nhiệm của 2 dạng công cụ sẽ được đề cập đến trong đồ án là công cụ triển khai sản phẩm (deployment tool) và công cụ quản lý cấu hình (configuration manager).


\thispagestyle{empty}