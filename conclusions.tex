\chapter*{Kết luận}
\addcontentsline{toc}{chapter}{Kết luận}

Tự động hóa hệ thống tuy không phải một vấn đề mới nhưng luôn là vấn đề đau đầu của những người quản trị hệ thống. Nhưng không giống như trước kia, ngày nay, chúng ta có thể sử dụng rất nhiều các công cụ để làm việc đó thay vì phải làm thủ công bằng các kịch bản truyền thống. Puppet, Chef hay Ansible là những công cụ cực kì hữu dụng mà bất cứ một nhà quản trị hệ thống nào hiện nay đều nên biết và sử dụng chúng trong công việc của mình.

Những kết quả của nghiên cứu này có thể là chưa sâu và bao quát hết được các tính năng của các công cụ đã giới thiệu. Nhưng đồ án này mang lại cái nhìn tổng quan và bao quát cho những ai chưa từng tiếp xúc với tự động hóa hệ thống, để họ có thể hiểu và áp dụng được nó vào thực tế công việc của từng người.

Người viết đồ án với mục đích là nghiên cứu để có thể triển khai trên thực tế các hoạt động tự động hóa cho hạ tầng các sản phẩm công nghệ đã học hỏi và rút ra được một số những kinh nghiệm quý cho bản thân. Trong tương lai, người viết sẽ cố gắng để hiểu sâu và thử nghiệm nhiều hơn nữa các công cụ này trong công việc hiện tại, nhằm nâng cao hiệu quả công việc của bản thân cũng như đồng nghiệp.

